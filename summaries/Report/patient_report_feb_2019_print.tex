\documentclass[]{article}
\usepackage{lmodern}
\usepackage{amssymb,amsmath}
\usepackage{ifxetex,ifluatex}
\usepackage{fixltx2e} % provides \textsubscript
\ifnum 0\ifxetex 1\fi\ifluatex 1\fi=0 % if pdftex
  \usepackage[T1]{fontenc}
  \usepackage[utf8]{inputenc}
\else % if luatex or xelatex
  \ifxetex
    \usepackage{mathspec}
  \else
    \usepackage{fontspec}
  \fi
  \defaultfontfeatures{Ligatures=TeX,Scale=MatchLowercase}
\fi
% use upquote if available, for straight quotes in verbatim environments
\IfFileExists{upquote.sty}{\usepackage{upquote}}{}
% use microtype if available
\IfFileExists{microtype.sty}{%
\usepackage{microtype}
\UseMicrotypeSet[protrusion]{basicmath} % disable protrusion for tt fonts
}{}
\usepackage[margin=1in]{geometry}
\usepackage{hyperref}
\hypersetup{unicode=true,
            pdftitle={Panacea Patient Report},
            pdfauthor={Martin Olarte},
            pdfborder={0 0 0},
            breaklinks=true}
\urlstyle{same}  % don't use monospace font for urls
\usepackage{longtable,booktabs}
\usepackage{graphicx,grffile}
\makeatletter
\def\maxwidth{\ifdim\Gin@nat@width>\linewidth\linewidth\else\Gin@nat@width\fi}
\def\maxheight{\ifdim\Gin@nat@height>\textheight\textheight\else\Gin@nat@height\fi}
\makeatother
% Scale images if necessary, so that they will not overflow the page
% margins by default, and it is still possible to overwrite the defaults
% using explicit options in \includegraphics[width, height, ...]{}
\setkeys{Gin}{width=\maxwidth,height=\maxheight,keepaspectratio}
\IfFileExists{parskip.sty}{%
\usepackage{parskip}
}{% else
\setlength{\parindent}{0pt}
\setlength{\parskip}{6pt plus 2pt minus 1pt}
}
\setlength{\emergencystretch}{3em}  % prevent overfull lines
\providecommand{\tightlist}{%
  \setlength{\itemsep}{0pt}\setlength{\parskip}{0pt}}
\setcounter{secnumdepth}{0}
% Redefines (sub)paragraphs to behave more like sections
\ifx\paragraph\undefined\else
\let\oldparagraph\paragraph
\renewcommand{\paragraph}[1]{\oldparagraph{#1}\mbox{}}
\fi
\ifx\subparagraph\undefined\else
\let\oldsubparagraph\subparagraph
\renewcommand{\subparagraph}[1]{\oldsubparagraph{#1}\mbox{}}
\fi

%%% Use protect on footnotes to avoid problems with footnotes in titles
\let\rmarkdownfootnote\footnote%
\def\footnote{\protect\rmarkdownfootnote}

%%% Change title format to be more compact
\usepackage{titling}

% Create subtitle command for use in maketitle
\newcommand{\subtitle}[1]{
  \posttitle{
    \begin{center}\large#1\end{center}
    }
}

\setlength{\droptitle}{-2em}

  \title{Panacea Patient Report}
    \pretitle{\vspace{\droptitle}\centering\huge}
  \posttitle{\par}
    \author{Martin Olarte}
    \preauthor{\centering\large\emph}
  \postauthor{\par}
      \predate{\centering\large\emph}
  \postdate{\par}
    \date{2/4/2019}


\begin{document}
\maketitle

{
\setcounter{tocdepth}{3}
\tableofcontents
}
\section{Abstract}\label{abstract}

This is an R Markdown document. Markdown is a simple formatting syntax
for authoring HTML, PDF, and MS Word documents. For more details on
using R Markdown see \url{http://rmarkdown.rstudio.com}.

When you click the \textbf{Knit} button a document will be generated
that includes both content as well as the output of any embedded R code
chunks within the document. You can embed an R code chunk like this:

\begin{center}\rule{0.5\linewidth}{\linethickness}\end{center}

\section{Treatment Description}\label{treatment-description}

This is an R Markdown document. Markdown is a simple formatting syntax
for authoring HTML, PDF, and MS Word documents. For more details on
using R Markdown see \url{http://rmarkdown.rstudio.com}.

When you click the \textbf{Knit} button a document will be generated
that includes both content as well as the output of any embedded R code
chunks within the document. You can embed an R code chunk like this:

\begin{center}\rule{0.5\linewidth}{\linethickness}\end{center}

\section{Patients}\label{patients}

\subsection{C1J}\label{c1j}

\subsubsection{Polarity Summary}\label{polarity-summary}

\begin{longtable}[]{@{}ccccccccc@{}}
\caption{C1-J Treatment Progress Summary}\tabularnewline
\toprule
Session & Start Date & End Date & Hours & Days & Starting Polarity &
Final Polarity & Change & Change per Treatment Hour\tabularnewline
\midrule
\endfirsthead
\toprule
Session & Start Date & End Date & Hours & Days & Starting Polarity &
Final Polarity & Change & Change per Treatment Hour\tabularnewline
\midrule
\endhead
1st & Sep 24 2018 & Oct 12 2018 & 113.00 & 17 & 214.3 & 116.1 & -98.2 &
-0.869\tabularnewline
2nd & Oct 22 2018 & Nov 03 2018 & 114.00 & 12 & 134.2 & 79.6 & -54.6 &
-0.479\tabularnewline
3rd & Nov 13 2018 & Nov 24 2018 & 124.10 & 11 & 98.7 & 34.5 & -64.2 &
-0.517\tabularnewline
4th & Dec 03 2018 & Dec 20 2018 & 162.03 & 15 & 35.6 & 7.9 & -27.7 &
-0.171\tabularnewline
5th & Feb 04 2019 & Feb 15 2019 & 99.50 & 11 & 34.6 & 15.6 & -19.0 &
-0.191\tabularnewline
\bottomrule
\end{longtable}

\begin{longtable}[]{@{}cccc@{}}
\caption{C1-J Retrogression Summary}\tabularnewline
\toprule
Break & Days & Increase in Polarity & Increase per Day without
Treatment\tabularnewline
\midrule
\endfirsthead
\toprule
Break & Days & Increase in Polarity & Increase per Day without
Treatment\tabularnewline
\midrule
\endhead
1st & 10 & 18.1 & 1.810\tabularnewline
2nd & 10 & 19.1 & 1.910\tabularnewline
3rd & 9 & 1.1 & 0.122\tabularnewline
4th & 46 & 26.7 & 0.580\tabularnewline
\bottomrule
\end{longtable}

\begin{center}\rule{0.5\linewidth}{\linethickness}\end{center}

\subsubsection{Treatment Sessions Plots}\label{treatment-sessions-plots}

\includegraphics[width=1\linewidth]{patient_report_feb_2019_print_files/figure-latex/C1J_hour_plot-1}

\begin{center}\rule{0.5\linewidth}{\linethickness}\end{center}

\subsubsection{Correlation Summary}\label{correlation-summary}

\begin{longtable}[]{@{}ccccc@{}}
\toprule
Session & Correlation Coefficient & P-value & CI Lower Bound & CI Upper
Bound\tabularnewline
\midrule
\endhead
1st & -0.9925 & 4.1575e-15 & -0.9974 & -0.9788\tabularnewline
2nd & -0.9910 & 2.8947e-08 & -0.9979 & -0.9608\tabularnewline
3rd & -0.9851 & 8.2273e-06 & -0.9974 & -0.9168\tabularnewline
4th & -0.8601 & 1.3025e-02 & -0.9790 & -0.3038\tabularnewline
5th & -0.9924 & 8.7450e-05 & -0.9992 & -0.9288\tabularnewline
\bottomrule
\end{longtable}

\begin{longtable}[]{@{}cccc@{}}
\toprule
Session & Adjusted R Squared & Residual Standard Error & Degrees of
Freedom\tabularnewline
\midrule
\endhead
1st & 0.9841 & 3.804 & 15\tabularnewline
2nd & 0.9797 & 2.581 & 8\tabularnewline
3rd & 0.9654 & 4.552 & 6\tabularnewline
4th & 0.6877 & 5.620 & 5\tabularnewline
5th & 0.9810 & 0.994 & 4\tabularnewline
\bottomrule
\end{longtable}

\begin{center}\rule{0.5\linewidth}{\linethickness}\end{center}

\subsubsection{Polarity through Time
Plot}\label{polarity-through-time-plot}

\includegraphics[width=1\linewidth]{patient_report_feb_2019_print_files/figure-latex/C1J_date_plot-1}

\begin{center}\rule{0.5\linewidth}{\linethickness}\end{center}

\subsection{B1S}\label{b1s}

\subsubsection{Polarity Summary}\label{polarity-summary-1}

\begin{longtable}[]{@{}ccccccccc@{}}
\caption{B1-S Treatment Progress Summary}\tabularnewline
\toprule
Session & Start Date & End Date & Hours & Days & Starting Polarity &
Final Polarity & Change & Change per Treatment Hour\tabularnewline
\midrule
\endfirsthead
\toprule
Session & Start Date & End Date & Hours & Days & Starting Polarity &
Final Polarity & Change & Change per Treatment Hour\tabularnewline
\midrule
\endhead
1st & Aug 21 2018 & Sep 08 2018 & 106.0 & 16 & 173.0 & 75.1 & -97.9 &
-0.924\tabularnewline
2nd & Sep 24 2018 & Oct 05 2018 & 59.0 & 11 & 91.2 & 20.9 & -70.3 &
-1.192\tabularnewline
3rd & Nov 13 2018 & Nov 17 2018 & 38.2 & 5 & 77.3 & 62.1 & -15.2 &
-0.398\tabularnewline
\bottomrule
\end{longtable}

\begin{longtable}[]{@{}cccc@{}}
\caption{B1-S Retrogression Summary}\tabularnewline
\toprule
Break & Days & Increase in Polarity & Increase per Day without
Treatment\tabularnewline
\midrule
\endfirsthead
\toprule
Break & Days & Increase in Polarity & Increase per Day without
Treatment\tabularnewline
\midrule
\endhead
1st & 16 & 16.1 & 1.006\tabularnewline
2nd & 39 & 56.4 & 1.446\tabularnewline
\bottomrule
\end{longtable}

\begin{center}\rule{0.5\linewidth}{\linethickness}\end{center}

\subsubsection{Treatment Sessions
Plots}\label{treatment-sessions-plots-1}

\includegraphics[width=1\linewidth]{patient_report_feb_2019_print_files/figure-latex/B1S_hour_plot-1}

\begin{center}\rule{0.5\linewidth}{\linethickness}\end{center}

\subsubsection{Correlation Summary}\label{correlation-summary-1}

\begin{longtable}[]{@{}ccccc@{}}
\caption{B1-S Treatment Correlation Summary}\tabularnewline
\toprule
Session & Correlation Coefficient & P-value & CI Lower Bound & CI Upper
Bound\tabularnewline
\midrule
\endfirsthead
\toprule
Session & Correlation Coefficient & P-value & CI Lower Bound & CI Upper
Bound\tabularnewline
\midrule
\endhead
1st & -0.9971 & 3.3275e-18 & -0.9990 & -0.9918\tabularnewline
2nd & -0.9867 & 3.2153e-09 & -0.9964 & -0.9517\tabularnewline
3rd & -0.9910 & 8.9542e-03 & -0.9998 & -0.6305\tabularnewline
\bottomrule
\end{longtable}

\begin{longtable}[]{@{}cccc@{}}
\toprule
Session & Adjusted R Squared & Residual Standard Error & Degrees of
Freedom\tabularnewline
\midrule
\endhead
1st & 0.9938 & 2.439 & 15\tabularnewline
2nd & 0.9709 & 3.901 & 10\tabularnewline
3rd & 0.9733 & 1.062 & 2\tabularnewline
\bottomrule
\end{longtable}

\begin{center}\rule{0.5\linewidth}{\linethickness}\end{center}

\subsubsection{Polarity through Time
Plot}\label{polarity-through-time-plot-1}

\includegraphics[width=1\linewidth]{patient_report_feb_2019_print_files/figure-latex/B1S_date_plot-1}

\begin{center}\rule{0.5\linewidth}{\linethickness}\end{center}

\subsection{L2E}\label{l2e}

\subsubsection{Polarity Summary}\label{polarity-summary-2}

\begin{longtable}[]{@{}ccccccccc@{}}
\caption{L2-E Treatment Progress Summary}\tabularnewline
\toprule
Session & Start Date & End Date & Hours & Days & Starting Polarity &
Final Polarity & Change & Change per Treatment Hour\tabularnewline
\midrule
\endfirsthead
\toprule
Session & Start Date & End Date & Hours & Days & Starting Polarity &
Final Polarity & Change & Change per Treatment Hour\tabularnewline
\midrule
\endhead
1st & Aug 21 2018 & Sep 01 2018 & 61.00 & 10 & 210.0 & 111.2 & -98.8 &
-1.620\tabularnewline
2nd & Sep 24 2018 & Oct 06 2018 & 64.00 & 12 & 123.1 & 35.1 & -88.0 &
-1.375\tabularnewline
3rd & Oct 22 2018 & Nov 03 2018 & 91.30 & 12 & 43.7 & 2.9 & -40.8 &
-0.447\tabularnewline
4th & Dec 03 2018 & Dec 15 2018 & 84.63 & 11 & 8.1 & 1.9 & -6.2 &
-0.073\tabularnewline
5th & Jan 08 2019 & Jan 19 2019 & 105.00 & 11 & 4.3 & 0.4 & -3.9 &
-0.037\tabularnewline
\bottomrule
\end{longtable}

\begin{longtable}[]{@{}cccc@{}}
\caption{L2-E Retrogression Summary}\tabularnewline
\toprule
Break & Days & Increase in Polarity & Increase per Day without
Treatment\tabularnewline
\midrule
\endfirsthead
\toprule
Break & Days & Increase in Polarity & Increase per Day without
Treatment\tabularnewline
\midrule
\endhead
1st & 23 & 11.9 & 0.517\tabularnewline
2nd & 16 & 8.6 & 0.538\tabularnewline
3rd & 30 & 5.2 & 0.173\tabularnewline
4th & 24 & 2.4 & 0.100\tabularnewline
\bottomrule
\end{longtable}

\begin{center}\rule{0.5\linewidth}{\linethickness}\end{center}

\subsubsection{Treatment Sessions
Plots}\label{treatment-sessions-plots-2}

\includegraphics[width=1\linewidth]{patient_report_feb_2019_print_files/figure-latex/L2E_hour_plot-1}

\begin{center}\rule{0.5\linewidth}{\linethickness}\end{center}

\subsubsection{Correlation Summary}\label{correlation-summary-2}

\begin{longtable}[]{@{}ccccc@{}}
\caption{L2-E Treatment Correlation Summary}\tabularnewline
\toprule
Session & Correlation Coefficient & P-value & CI Lower Bound & CI Upper
Bound\tabularnewline
\midrule
\endfirsthead
\toprule
Session & Correlation Coefficient & P-value & CI Lower Bound & CI Upper
Bound\tabularnewline
\midrule
\endhead
1st & -0.9965 & 5.0705e-11 & -0.9991 & -0.9861\tabularnewline
2nd & -0.9986 & 3.6131e-14 & -0.9996 & -0.9950\tabularnewline
3rd & -0.9628 & 3.1561e-05 & -0.9924 & -0.8283\tabularnewline
4th & -0.9722 & 1.1506e-03 & -0.9971 & -0.7611\tabularnewline
\bottomrule
\end{longtable}

\begin{longtable}[]{@{}cccc@{}}
\toprule
Session & Adjusted R Squared & Residual Standard Error & Degrees of
Freedom\tabularnewline
\midrule
\endhead
1st & 0.9923 & 2.8850 & 9\tabularnewline
2nd & 0.9970 & 1.5200 & 10\tabularnewline
3rd & 0.9165 & 4.4740 & 7\tabularnewline
4th & 0.9314 & 0.6101 & 4\tabularnewline
\bottomrule
\end{longtable}

\begin{center}\rule{0.5\linewidth}{\linethickness}\end{center}

\subsubsection{Polarity through Time
Plot}\label{polarity-through-time-plot-2}

\includegraphics[width=1\linewidth]{patient_report_feb_2019_print_files/figure-latex/L2E_date_plot-1}

\begin{center}\rule{0.5\linewidth}{\linethickness}\end{center}

\subsection{H1A}\label{h1a}

\subsubsection{Polarity Summary}\label{polarity-summary-3}

\begin{longtable}[]{@{}ccccccccc@{}}
\caption{H1-A Treatment Progress Summary}\tabularnewline
\toprule
Session & Start Date & End Date & Hours & Days & Starting Polarity &
Final Polarity & Change & Change per Treatment Hour\tabularnewline
\midrule
\endfirsthead
\toprule
Session & Start Date & End Date & Hours & Days & Starting Polarity &
Final Polarity & Change & Change per Treatment Hour\tabularnewline
\midrule
\endhead
1st & Aug 23 2018 & Sep 01 2018 & 31.0 & 8 & 63.0 & 4.3 & -58.7 &
-1.894\tabularnewline
2nd & Sep 24 2018 & Oct 06 2018 & 32.0 & 8 & 6.2 & 1.4 & -4.8 &
-0.150\tabularnewline
3rd & Nov 01 2018 & Nov 01 2018 & 2.0 & 1 & 3.4 & 3.4 & 0.0 &
0.000\tabularnewline
4th & Dec 03 2018 & Dec 14 2018 & 18.0 & 5 & 12.3 & 8.6 & -3.7 &
-0.206\tabularnewline
5th & Jan 08 2019 & Jan 19 2019 & 38.8 & 10 & 10.3 & 2.7 & -7.6 &
-0.196\tabularnewline
\bottomrule
\end{longtable}

\begin{longtable}[]{@{}cccc@{}}
\caption{H1-A Retrogression Summary}\tabularnewline
\toprule
Break & Days & Increase in Polarity & Increase per Day without
Treatment\tabularnewline
\midrule
\endfirsthead
\toprule
Break & Days & Increase in Polarity & Increase per Day without
Treatment\tabularnewline
\midrule
\endhead
1st & 23 & 1.9 & 0.083\tabularnewline
2nd & 26 & 2.0 & 0.077\tabularnewline
3rd & 32 & 8.9 & 0.278\tabularnewline
4th & 25 & 1.7 & 0.068\tabularnewline
\bottomrule
\end{longtable}

\begin{center}\rule{0.5\linewidth}{\linethickness}\end{center}

\subsubsection{Treatment Sessions
Plots}\label{treatment-sessions-plots-3}

\includegraphics[width=1\linewidth]{patient_report_feb_2019_print_files/figure-latex/H1A_hour_plot-1}

\begin{center}\rule{0.5\linewidth}{\linethickness}\end{center}

\subsubsection{Correlation Summary}\label{correlation-summary-3}

\begin{longtable}[]{@{}ccccc@{}}
\caption{H1-A Treatment Correlation Summary}\tabularnewline
\toprule
Session & Correlation Coefficient & P-value & CI Lower Bound & CI Upper
Bound\tabularnewline
\midrule
\endfirsthead
\toprule
Session & Correlation Coefficient & P-value & CI Lower Bound & CI Upper
Bound\tabularnewline
\midrule
\endhead
1st & -0.9936 & 6.6280e-07 & -0.9989 & -0.9634\tabularnewline
2nd & -0.9837 & 1.8010e-06 & -0.9967 & -0.9216\tabularnewline
3rd & NA & NA & NA & NA\tabularnewline
4th & -0.9800 & -9.8000e-01 & -0.9800 & -0.9800\tabularnewline
5th & -0.9003 & -9.0030e-01 & -0.9003 & -0.9003\tabularnewline
\bottomrule
\end{longtable}

\begin{longtable}[]{@{}cccc@{}}
\toprule
Session & Adjusted R Squared & Residual Standard Error & Degrees of
Freedom\tabularnewline
\midrule
\endhead
1st & 0.9850 & 2.6320 & 6\tabularnewline
2nd & 0.9630 & 0.3196 & 7\tabularnewline
3rd & NA & NA & NA\tabularnewline
4th & 0.9210 & 0.5307 & 1\tabularnewline
5th & 0.7835 & 1.1690 & 7\tabularnewline
\bottomrule
\end{longtable}

\begin{center}\rule{0.5\linewidth}{\linethickness}\end{center}

\subsubsection{Polarity through Time
Plot}\label{polarity-through-time-plot-3}

\includegraphics[width=1\linewidth]{patient_report_feb_2019_print_files/figure-latex/H1A_date_plot-1}

\begin{center}\rule{0.5\linewidth}{\linethickness}\end{center}

\subsection{L3K}\label{l3k}

\subsubsection{Polarity Summary}\label{polarity-summary-4}

\begin{longtable}[]{@{}ccccccccc@{}}
\caption{L3-K Treatment Progress Summary}\tabularnewline
\toprule
Session & Start Date & End Date & Hours & Days & Starting Polarity &
Final Polarity & Change & Change per Treatment Hour\tabularnewline
\midrule
\endfirsthead
\toprule
Session & Start Date & End Date & Hours & Days & Starting Polarity &
Final Polarity & Change & Change per Treatment Hour\tabularnewline
\midrule
\endhead
1st & Oct 29 2018 & Nov 03 2018 & 52.00 & 6 & 237.2 & 197.6 & -39.6 &
-0.762\tabularnewline
2nd & Dec 03 2018 & Dec 15 2018 & 93.98 & 11 & 209.3 & 167.1 & -42.2 &
-0.449\tabularnewline
3rd & Jan 14 2019 & Jan 19 2019 & 43.00 & 6 & 180.3 & 148.7 & -31.6 &
-0.735\tabularnewline
4th & Feb 11 2019 & Feb 14 2019 & 31.50 & 4 & 159.3 & 146.2 & -13.1 &
-0.416\tabularnewline
\bottomrule
\end{longtable}

\begin{longtable}[]{@{}cccc@{}}
\caption{L3-K Retrogression Summary}\tabularnewline
\toprule
Break & Days & Increase in Polarity & Increase per Day without
Treatment\tabularnewline
\midrule
\endfirsthead
\toprule
Break & Days & Increase in Polarity & Increase per Day without
Treatment\tabularnewline
\midrule
\endhead
1st & 30 & 11.7 & 0.390\tabularnewline
2nd & 30 & 13.2 & 0.440\tabularnewline
3rd & 23 & 10.6 & 0.461\tabularnewline
\bottomrule
\end{longtable}

\begin{center}\rule{0.5\linewidth}{\linethickness}\end{center}

\subsubsection{Treatment Sessions
Plots}\label{treatment-sessions-plots-4}

\includegraphics[width=1\linewidth]{patient_report_feb_2019_print_files/figure-latex/L3K_hour_plot-1}

\begin{center}\rule{0.5\linewidth}{\linethickness}\end{center}

\subsubsection{Correlation Summary}\label{correlation-summary-4}

\begin{longtable}[]{@{}ccccc@{}}
\caption{L3-K Treatment Correlation Summary}\tabularnewline
\toprule
Session & Correlation Coefficient & P-value & CI Lower Bound & CI Upper
Bound\tabularnewline
\midrule
\endfirsthead
\toprule
Session & Correlation Coefficient & P-value & CI Lower Bound & CI Upper
Bound\tabularnewline
\midrule
\endhead
1st & -0.9984 & 3.7021e-06 & -0.9998 & -0.9850\tabularnewline
2nd & -0.9884 & 2.7951e-05 & -0.9984 & -0.9201\tabularnewline
3rd & -0.9980 & 5.9017e-06 & -0.9998 & -0.9811\tabularnewline
4th & -0.9848 & 1.5221e-02 & -0.9997 & -0.4425\tabularnewline
\bottomrule
\end{longtable}

\begin{longtable}[]{@{}cccc@{}}
\toprule
Session & Adjusted R Squared & Residual Standard Error & Degrees of
Freedom\tabularnewline
\midrule
\endhead
1st & 0.9961 & 0.9709 & 4\tabularnewline
2nd & 0.9722 & 2.6480 & 5\tabularnewline
3rd & 0.9950 & 0.8289 & 4\tabularnewline
4th & 0.9547 & 1.2180 & 2\tabularnewline
\bottomrule
\end{longtable}

\begin{center}\rule{0.5\linewidth}{\linethickness}\end{center}

\subsubsection{Polarity through Time
Plot}\label{polarity-through-time-plot-4}

\includegraphics[width=1\linewidth]{patient_report_feb_2019_print_files/figure-latex/L3K_date_plot-1}

\begin{center}\rule{0.5\linewidth}{\linethickness}\end{center}

\subsection{G1D}\label{g1d}

\subsubsection{Polarity Summary}\label{polarity-summary-5}

\begin{longtable}[]{@{}ccccccccc@{}}
\caption{G1-D Treatment Progress Summary}\tabularnewline
\toprule
Session & Start Date & End Date & Hours & Days & Starting Polarity &
Final Polarity & Change & Change per Treatment Hour\tabularnewline
\midrule
\endfirsthead
\toprule
Session & Start Date & End Date & Hours & Days & Starting Polarity &
Final Polarity & Change & Change per Treatment Hour\tabularnewline
\midrule
\endhead
1st & Sep 24 2018 & Sep 26 2018 & 16.0 & 3 & 9.8 & 3.9 & -5.9 &
-0.369\tabularnewline
2nd & Nov 01 2018 & Nov 03 2018 & 11.5 & 3 & 6.3 & 2.8 & -3.5 &
-0.304\tabularnewline
3rd & Dec 11 2018 & Dec 13 2018 & 18.3 & 3 & 3.9 & 1.3 & -2.6 &
-0.142\tabularnewline
4th & Jan 15 2019 & Jan 17 2019 & 16.0 & 3 & 2.8 & 1.7 & -1.1 &
-0.069\tabularnewline
\bottomrule
\end{longtable}

\begin{longtable}[]{@{}cccc@{}}
\caption{G1-D Retrogression Summary}\tabularnewline
\toprule
Break & Days & Increase in Polarity & Increase per Day without
Treatment\tabularnewline
\midrule
\endfirsthead
\toprule
Break & Days & Increase in Polarity & Increase per Day without
Treatment\tabularnewline
\midrule
\endhead
1st & 36 & 2.4 & 0.067\tabularnewline
2nd & 38 & 1.1 & 0.029\tabularnewline
3rd & 33 & 1.5 & 0.045\tabularnewline
\bottomrule
\end{longtable}

\begin{center}\rule{0.5\linewidth}{\linethickness}\end{center}

\subsubsection{Treatment Sessions
Plots}\label{treatment-sessions-plots-5}

\includegraphics[width=1\linewidth]{patient_report_feb_2019_print_files/figure-latex/G1D_hour_plot-1}

\begin{center}\rule{0.5\linewidth}{\linethickness}\end{center}

\subsubsection{Correlation Summary}\label{correlation-summary-5}

\begin{longtable}[]{@{}ccccc@{}}
\caption{G1-D Treatment Correlation Summary}\tabularnewline
\toprule
Session & Correlation Coefficient & P-value & CI Lower Bound & CI Upper
Bound\tabularnewline
\midrule
\endfirsthead
\toprule
Session & Correlation Coefficient & P-value & CI Lower Bound & CI Upper
Bound\tabularnewline
\midrule
\endhead
1st & -0.9964 & 3.5556e-03 & -0.9999 & -0.8353\tabularnewline
2nd & -0.9846 & 1.5399e-02 & -0.9997 & -0.4378\tabularnewline
3rd & -0.9740 & 2.6046e-02 & -0.9995 & -0.2012\tabularnewline
4th & -0.9978 & 4.2347e-02 & NA & NA\tabularnewline
\bottomrule
\end{longtable}

\begin{longtable}[]{@{}cccc@{}}
\toprule
Session & Adjusted R Squared & Residual Standard Error & Degrees of
Freedom\tabularnewline
\midrule
\endhead
1st & 0.9894 & 0.27590 & 2\tabularnewline
2nd & 0.9542 & 0.31530 & 2\tabularnewline
3rd & 0.9229 & 0.32820 & 2\tabularnewline
4th & 0.9912 & 0.05345 & 1\tabularnewline
\bottomrule
\end{longtable}

\begin{center}\rule{0.5\linewidth}{\linethickness}\end{center}

\subsubsection{Polarity through Time
Plot}\label{polarity-through-time-plot-5}

\includegraphics[width=1\linewidth]{patient_report_feb_2019_print_files/figure-latex/G1D_date_plot-1}

\begin{center}\rule{0.5\linewidth}{\linethickness}\end{center}

\subsection{L1J}\label{l1j}

\subsubsection{Polarity Summary}\label{polarity-summary-6}

\begin{longtable}[]{@{}ccccccccc@{}}
\caption{L1-J Treatment Progress Summary}\tabularnewline
\toprule
Session & Start Date & End Date & Hours & Days & Starting Polarity &
Final Polarity & Change & Change per Treatment Hour\tabularnewline
\midrule
\endfirsthead
\toprule
Session & Start Date & End Date & Hours & Days & Starting Polarity &
Final Polarity & Change & Change per Treatment Hour\tabularnewline
\midrule
\endhead
1st & Aug 13 2018 & Sep 07 2018 & 98.5 & 19 & 187.3 & 15.9 & -171.4 &
-1.74\tabularnewline
\bottomrule
\end{longtable}

\begin{center}\rule{0.5\linewidth}{\linethickness}\end{center}

\subsubsection{Treatment Sessions
Plots}\label{treatment-sessions-plots-6}

\includegraphics[width=1\linewidth]{patient_report_feb_2019_print_files/figure-latex/L1J_hour_plot-1}

\begin{center}\rule{0.5\linewidth}{\linethickness}\end{center}

\subsubsection{Correlation Summary}\label{correlation-summary-6}

\begin{longtable}[]{@{}ccccc@{}}
\caption{L1-J Treatment Correlation Summary}\tabularnewline
\toprule
Session & Correlation Coefficient & P-value & CI Lower Bound & CI Upper
Bound\tabularnewline
\midrule
\endfirsthead
\toprule
Session & Correlation Coefficient & P-value & CI Lower Bound & CI Upper
Bound\tabularnewline
\midrule
\endhead
1st & -0.996 & 3.1020e-18 & -0.9986 & -0.9891\tabularnewline
\bottomrule
\end{longtable}

\begin{longtable}[]{@{}cccc@{}}
\toprule
Session & Adjusted R Squared & Residual Standard Error & Degrees of
Freedom\tabularnewline
\midrule
\endhead
1st & 0.9916 & 5.467 & 16\tabularnewline
\bottomrule
\end{longtable}

\begin{center}\rule{0.5\linewidth}{\linethickness}\end{center}

\subsubsection{Polarity through Time
Plot}\label{polarity-through-time-plot-6}

\includegraphics[width=1\linewidth]{patient_report_feb_2019_print_files/figure-latex/L1J_date_plot-1}

\begin{center}\rule{0.5\linewidth}{\linethickness}\end{center}

\section{Notes}\label{notes}

Regarding \emph{Total Treatment Hours vs Polarity} plots:

\begin{itemize}
\tightlist
\item
  The linear regression models were fitted using the
  \texttt{geom\_smooth(method\ =\ "lm",\ y\ \textasciitilde{}\ x)}
  function in R.
\end{itemize}

Regarding \emph{Polarity through Time} plots:

\begin{itemize}
\tightlist
\item
  The dashed linear segments represent the overall change in polarity
  during breaks. However, they do not accurately model the nature of
  these changes due to a lack of intermediary measurements.
\end{itemize}

Regarding \emph{Correlation Summaries}:

\begin{itemize}
\item
  The \textbf{correlation coefficients} (R) refer to the
  \href{https://statistics.laerd.com/statistical-guides/pearson-correlation-coefficient-statistical-guide.php}{Pearson
  product-moment correlation coefficients}. According to the
  \href{https://en.wikipedia.org/wiki/Cauchy\%E2\%80\%93Schwarz_inequality}{Cauchy-Schwarz
  inequality} it has a value between 1 and -1, where 1 represents a
  total positive linear correlation, 0 no linear correlation, and -1 a
  total negative correlation.
\item
  The \textbf{p-values} refer to the
  \href{https://en.wikipedia.org/wiki/P-value\#Definition_and_interpretation}{asymptotic
  significance of the R estimates}; for each specific linear model, they
  describe the probability that the correlation estimate would be
  greater than or equal to the actual correlation, assuming that the
  true correlation is equal to 0 (true null hypothesis). The p-values
  were calculated using a t-distribution with n -- 2 degrees of freedom,
  and a significance level of 0.05 was used. Thus, a sample with a
  p-value less than 0.05 provides reasonable evidence to support the
  alternative hypothesis (true correlation is not equal to 0).
\item
  The \textbf{CI bounds} refer to a 95\% asymptotic confidence interval
  for the correlation coefficient (R) estimates based on
  \href{https://en.wikipedia.org/wiki/Fisher_transformation}{Fisher's Z
  transform}, on cases of at least 4 complete pairs of observations
  (polarity value and therapy hours).
\item
  The \textbf{adjusted R-squared} values were calculated using the
  \href{https://www-bcf.usc.edu/~gareth/ISL/ISLR\%20First\%20Printing.pdf\#page=227}{unbiased
  correlation coefficients and number of observations in each sample}.
  They represent the percentage of variation explained by only the
  independent variable (treatment hours) that actually affect the
  dependent variable (polarity). Thus, this value increases only if a
  new term improves the model more than would be expected by chance, and
  decreases when a predictor improves the model by less than expected by
  chance.
\item
  The \textbf{residual standard error}
  \href{http://statweb.stanford.edu/~susan/courses/s60/split/node60.html}{(RSE)}
  values were calculated by taking the positive square root of the mean
  square error of the residuals for each linear model. The smaller the
  RSE, the better the model fits the data.
\item
  The \textbf{degrees of freedom} refer to the number of independent
  pieces of information on which the correlation estimate is based,
  \href{http://onlinestatbook.com/2/estimation/df.html}{according to
  David M. Lane}, associate professor at Rice University. They were
  calculated by subtracting the number of parameters estimated (2) from
  the number of observations within each linear model (sample sizes).
\end{itemize}

\begin{center}\rule{0.5\linewidth}{\linethickness}\end{center}

\section{General Conclusions}\label{general-conclusions}

This is an R Markdown document. Markdown is a simple formatting syntax
for authoring HTML, PDF, and MS Word documents. For more details on
using R Markdown see \url{http://rmarkdown.rstudio.com}.

When you click the \textbf{Knit} button a document will be generated
that includes both content as well as the output of any embedded R code
chunks within the document. You can embed an R code chunk like this:

\begin{center}\rule{0.5\linewidth}{\linethickness}\end{center}

\section{About Panacea}\label{about-panacea}

This is an R Markdown document. Markdown is a simple formatting syntax
for authoring HTML, PDF, and MS Word documents. For more details on
using R Markdown see \url{http://rmarkdown.rstudio.com}.

When you click the \textbf{Knit} button a document will be generated
that includes both content as well as the output of any embedded R code
chunks within the document. You can embed an R code chunk like this:

\begin{center}\rule{0.5\linewidth}{\linethickness}\end{center}


\end{document}
